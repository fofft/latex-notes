\documentclass[8pt]{report}
\usepackage[T1]{fontenc}
\usepackage{algorithm}
\usepackage{algpseudocode}
\usepackage{amsfonts}
\usepackage{amsmath}
\usepackage{amssymb}
\usepackage{enumitem}
\usepackage{fixltx2e}
\usepackage{framed}
\usepackage[empty]{fullpage}
\usepackage{graphicx}
\usepackage{hyperref}
\usepackage{mathtools}
\usepackage{parskip}
\usepackage{pdfpages}
\addtolength{\oddsidemargin}{-0.25in}
\addtolength{\evensidemargin}{-0.25in}
\addtolength{\textwidth}{0.5in}
\title{Class No. - Class Name}
\author{Student Name\\[1cm]{\small Professor: Professor Name}}
\date{Fall 2017}
\begin{document}

% Lazy floor function
\newcommand{\floor}[1]{\lfloor #1 \rfloor}

% Add in a box for a quick way to see where classes were started.
% This will also include an entry in the table of contents for the date.
\newcommand{\StartOfClass}[1] {
	\addcontentsline{toc}{subsection}{\protect\numberline{}#1}
  	\noindent\fbox{%
    \parbox{\textwidth}{%
      \begin{center}
        \textbf{Start of class #1}
      \end{center}
    }%
  }
}

% Adds any handouts inline with the rest of the document.
% Adds handout to the table of contents.
\newcommand{\Handout}[1] {
	\addcontentsline{toc}{section}{\protect\numberline{}#1}
	\includepdf[pages={-}]{#1}
}

\maketitle
\newpage
\tableofcontents
\newpage
\chapter{Class Overview}

\section{Class Logistics}

\StartOfClass{2017.08.29}

\end{document}
